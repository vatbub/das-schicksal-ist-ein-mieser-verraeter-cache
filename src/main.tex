%! suppress = Ellipsis
% Preamble
\documentclass[12pt]{article}

\usepackage[utf8]{inputenc}
\usepackage[T1]{fontenc}

% Translation
\newcommand{\translationDefaultCharacterDescription}{Keine Charakterbeschreibung angegeben.}
\newcommand{\translationDecisionLabel}{Entscheidung}
\newcommand{\translationDecisionPrincipalSuffix}{trifft die Entscheidung}
\newcommand{\translationOptionLabel}{Option}
\newcommand{\translationEndOfBranch}{Ende von}
\newcommand{\translationConditionalLabel}{Bedingung}
\newcommand{\translationBeginningOfConditional}{Wenn}
\newcommand{\translationContinued}{Fortsetzung}

\newcommand{\translationWrittenBy}{Geschrieben von:}
\newcommand{\translationBasedOn}{Basierend auf:}


% Read preamble
% Document font
\usepackage{courier}

% Margins
\usepackage[a4paper, top=2.5cm, bottom=2cm, left=2.5cm, right=2.5cm]{geometry}
\setlength{\parindent}{0pt}
\setlength{\parskip}{\baselineskip}

% Packages
\usepackage{glossaries}
\usepackage{changepage}
\usepackage{xparse}
\usepackage{titling}
\usepackage{array}
\usepackage{enumerate}
\usepackage[shortlabels]{enumitem}
\usepackage{titlesec}
\usepackage{calc}

% For title slug
\newcolumntype{P}[1]{>{\centering\arraybackslash}p{#1}}

% Section header formatting
%! suppress = MissingLabel
\titleformat{\section}{}{}{0em}{\MakeUppercase}

\makeglossaries

% Appearance of characters
\newcommand{\character}[2][\translationDefaultCharacterDescription]{\newacronym{#2}{#2}{#1}}
\newacronymstyle{screenplay-characters}
{%
    \GlsUseAcrEntryDispStyle{short-long}%
}
{%
    \GlsUseAcrStyleDefs{short-long}%
    \renewcommand{\firstacronymfont}[1]{\MakeUppercase{##1}}%
    %
    % No case change, singular first use:
    \renewcommand*{\genacrfullformat}[2]{%
        \firstacronymfont{\glsentryshort{##1}}%
    }%
    % First letter upper case, singular first use:
    \renewcommand*{\Genacrfullformat}[2]{%
        \firstacronymfont{\Glsentryshort{##1}}%
    }%
    % No case change, plural first use:
    \renewcommand*{\genplacrfullformat}[2]{%
        \firstacronymfont{\glsentryshortpl{##1}}%
    }%
    % First letter upper case, plural first use:
    \renewcommand*{\Genplacrfullformat}[2]{%
        \firstacronymfont{\Glsentryshortpl{##1}}%
    }%
}

\setacronymstyle{screenplay-characters}

% Elements of a scene
\newenvironment{scene}[4][\unskip]{
    \textbf{\uppercase{#1}}


    \section{#2 #3 -- #4}\label{sec:scene-#3}
    }{}
\newenvironment{scenedescription}{%
    \begin{flushleft}
    }{
    \end{flushleft}
}

\newcommand{\standaloneactionline}[1]{#1}
\newcommand{\actionline}[1]{\standaloneactionline{#1}}
\newcommand{\dialogactionline}[2]{
    \end{flushleft}
    \end{adjustwidth}
    #1
    \begin{adjustwidth}{6.5cm}{0cm}
        \uppercase{#2}~(\translationContinued)
    \end{adjustwidth}
    \begin{adjustwidth}{2.5cm}{0cm}
        \begin{flushleft}
}

% Dialog elements
\NewDocumentEnvironment{dialog} { o m } {
    \begin{adjustwidth}{6.5cm}{0cm}
        \uppercase{#2}\IfValueT{#1}{\uppercase{~(#1)}}
    \end{adjustwidth}
    \renewcommand{\actionline}[1]{\dialogactionline{##1}{#2}}
    \begin{adjustwidth}{2.5cm}{0cm}
        \begin{flushleft}
        }{
        \end{flushleft}
    \end{adjustwidth}
\renewcommand{\actionline}[1]{\standaloneactionline{#1}}
}

% Other helpful stuff
\newcommand{\direction}[1]{\emph{(#1)}}
\newcommand{\pause}[0]{\direction{Pause}}
\newcommand{\beat}[0]{\direction{Beat}}

% Decision elements
\newlist{optionlist}{enumerate}{3}
\setlist[optionlist]{
    font=\bfseries,
    align=left
}
\setlist[optionlist, 1]{
    label=\translationOptionLabel~\arabic*:,
    ref=\arabic*
}
\setlist[optionlist, 2]{
    label=\translationOptionLabel~\theoptionlisti.\arabic*:,
    ref=\theoptionlisti.\arabic*
}
\setlist[optionlist, 3]{
    label=\translationOptionLabel~\theoptionlisti.\theoptionlistii.\arabic*:,
    ref=\theoptionlisti.\theoptionlistii.\arabic*
}

\newcounter{DecisionEnvironmentDepth}
\setcounter{DecisionEnvironmentDepth}{0}
\newcounter{decision}
\newenvironment{decision}[2]{
    \refstepcounter{decision}
    \addtocounter{DecisionEnvironmentDepth}{1}% increment depth
    \expandafter\newcommand\csname DecisionLabel\the\value{DecisionEnvironmentDepth} \endcsname{#1}
    \label{#1}
    \par\textbf{\translationDecisionLabel~\thedecision:}~#2
    \begin{optionlist}
    }{
    \end{optionlist}
    \addtocounter{DecisionEnvironmentDepth}{-1}% decrement depth
}

\newlength{\enumerateIndent}
\setlength{\enumerateIndent}{-1cm}
\newcounter{OptionEnvironmentDepth}
\setcounter{OptionEnvironmentDepth}{0}
\newcommand{\currentOptionLabel}{}
\newenvironment{option}[2]{
    \renewcommand{\currentOptionLabel}{#1}
    \addtocounter{OptionEnvironmentDepth}{1}% increment depth
    \item \label{#1} #2
    \begin{adjustwidth}{\enumerateIndent}{0cm*\value{OptionEnvironmentDepth}}
    }{
    \end{adjustwidth}
    \textit{(\translationEndOfBranch~\translationDecisionLabel~\ref{\csname DecisionLabel\the\value{DecisionEnvironmentDepth} \endcsname},~\translationOptionLabel~\ref{\currentOptionLabel})}
    \addtocounter{OptionEnvironmentDepth}{-1}% decrement depth
}

\newcounter{ConditionalEnvironmentDepth}
\setcounter{ConditionalEnvironmentDepth}{0}
\newcounter{conditional}
\newenvironment{conditional}[2]{
    \refstepcounter{conditional}
    \addtocounter{ConditionalEnvironmentDepth}{1}% increment depth
    \expandafter\newcommand\csname ConditionalCounter\the\value{ConditionalEnvironmentDepth} \endcsname{\theconditional}
    \textbf{\translationConditionalLabel~\theconditional:} \translationBeginningOfConditional~\translationDecisionLabel~\ref{#1}, \translationOptionLabel~\ref{#2}:
        }{
    \textit{(\translationEndOfBranch~\translationConditionalLabel~\csname ConditionalCounter\the\value{ConditionalEnvironmentDepth} \endcsname.)}
    \addtocounter{ConditionalEnvironmentDepth}{-1}% decrement depth
}


% Characters
\character{Hazel}
\character{Augustus}
\character{Hazels Mutter}
\character{Dr. Jim}
\character{Patrick}
\character{Isaac}
\character{Monica}


% Title
\title{Das Schicksal ist ein mieser Verräter}
\author{Frederik Kammel}
\newcommand{\BasedOn}{Das Schicksal ist ein mieser Verräter, John Green}
\newcommand{\TitleSlug}{Stell dir vor, dein Leben liegt in Trümmern. \newline Der letzte Kuss ist Jahre her.}

% Document
\begin{document}
    \fontfamily{pcr}\selectfont

\begin{titlepage}
    \centering
    \vglue6cm
    „\MakeUppercase{\thetitle}“

    \vspace{2cm}
    \translationWrittenBy

    \theauthor

    \vspace{2cm}
    \translationBasedOn

    \BasedOn

    \ifdefined\TitleSlug
    \vspace{2cm}

    \begin{adjustwidth}{1cm}{1cm}
        \begin{center}
            \begin{tabular}{ P{13cm} }
                \hline
                \textit{\TitleSlug} \\
                \hline
            \end{tabular}
        \end{center}
    \end{adjustwidth}
    \fi
\end{titlepage}



    \begin{scene}[fade in]{innen}{Prolog}{Tagsüber}
        \begin{dialog}[V.O.]{Hazel}
            Depression ist eine Nebenwirkung von Krebs.
            So steht es zumindest in jeder Krebs-Broschüre oder auf jeder Website im Internet.
            In Wirklichkeit sind Depressionen aber eine Nebenwirkung des Sterbens.
            Krebs ist auch eine Nebenwirkung des Sterbens.

            \pause

            Meine Mutter war der festen Überzeugung, dass ich eine Therapie brauchte, denn ich saß die ganze Zeit in
            meinem Zimmer, verließ kaum das Haus, las immer wieder das selbe Buch, aß wenig und verbrachte den großen
            Teil meiner reichlichen Zeit, um über den Tod nachzudenken.
            So gingen wir zu \gls{Dr. Jim}, der meiner Mutter bestätigte, dass ich bis zum Hals in einer lähmenden und absolut
            klinischen Depression steckte und dass meine Medikamente neu eingestellt werden müssten und ich außerdem
            einmal die Woche eine Selbsthilfegruppe besuchen sollte.
        \end{dialog}
    \end{scene}

    \begin{scene}[cut to]{innen}{Selbsthilfegruppe 1}{Nachmittag}
        \begin{scenedescription}
            Eine Gruppe Jugendlicher sitzt in einem Stuhlkreis in einem Kellerraum einer Kirche, unter ihnen ein älterer Betreuer, \gls{Patrick}.
            In der Mitte des Stuhlkreises ist ein Teppich ausgerollt, auf dem eine Abbildung Jesu-Christi zu sehen ist.
            Auf der Abbildung ist sein Herz deutlich zu sehen.
            Unter den Jugendlichen sitzen:

            \gls{Hazel}.
            Sie hat Lungenkrebs und zieht daher ständig eine Sauerstoffflasche hinter sich her.

            \gls{Isaac}.
            Er hat Augenkrebs, weswegen bei ihm ein Auge durch ein Glasauge ersetzt wurde.

            \gls{Augustus}.
            Er hatte Knochenkrebs.
            Im Rahmen seiner Behandlung wurde ihm sein rechtes Bein amputiert, weshalb er dort eine Prothese trägt.

            Die Stimmung ist sehr deprimierend, weswegen jeder nur nach explizitem Aufruf von \gls{Patrick} spricht, ansonsten ist es still.
        \end{scenedescription}

        \actionline{\gls{Augustus} starrt die ganze Zeit \gls{Hazel} an.}

        \begin{dialog}[betend]{Patrick}
            Gott, gib mir die Gelassenheit, die Dinge hinzunehmen, die ich nicht ändern kann, den Mut, die Dinge zu ändern,
            die ich ändern kann, und die Weisheit, das eine vom anderen zu unterscheiden.
        \end{dialog}

        \begin{dialog}[Normal]{Patrick}
            Nun, wir sind alle hier, buchstäblich im Herzen Jesu, weil jeder von uns eine Krebsgeschichte hat.
            Ich zum Beispiel hatte als Kind Hodenkrebs.
            Meine Eltern und meine Ärzte dachten, ich würde von ihnen gehen.

            Aber Gott hat mich gerettet und so bin ich nun hier, bei euch.
            Um euch zu helfen.

            Wir haben ein paar neue Gesichter hier, vielleicht stellt ihr euch alle mal vor.
            Isaac, möchtest du heute anfangen?
            Ich weiß, dass dir in der kommenden Woche eine große Herausforderung bevorsteht.
        \end{dialog}

        \begin{dialog}{Isaac}
            Ok. \direction{Räuspert sich}

            Ich heiße \gls{Isaac}.
            Ich bin siebzehn.
            Ich habe Augenkrebs.
            Mir wurde ja schon ein Auge entfernt, aber am Montag werde ich wieder operiert und das zweite Auge wird herausgenommen.
            Danach bin ich blind, das ist klar.
            Ich will mich auch nicht beschweren oder so, denn viele von euch hat es ja viel schlimmer erwischt.
            Aber, na ja, blind werden ist auch irgendwie scheiße.
            Zum Glück habe ich aber eine verdammt heiße Freundin, \gls{Monica}.
            Weiß auch nicht, wie ich die verdient habe.

            Und ich habe tolle Freunde, wie \gls{Augustus} Waters, die mir beistehen.

            \actionline{\gls{Isaac} nickt \gls{Augustus} zu, der neben ihm sitzt.}

            Tja, so sieht's aus.
            Danke.
        \end{dialog}

        \begin{dialog}{Patrick}
            Wir sind für dich da, \gls{Isaac}.
            Sagen wir es ihm, Leute.
        \end{dialog}

        \begin{dialog}{Alle im Chor}
            Wir sind für dich da \gls{Isaac}.
        \end{dialog}

        \begin{dialog}{Patrick}
            Hazel, wie sieht es mit dir aus?
        \end{dialog}

        \begin{dialog}{Hazel}
            Oh, äh, ... Ich heiße Hazel.
            Ich bin sechzehn.
            Ursprünglich hatte ich Schilddrüsenkrebs, Stadium IV, aber mit umfänglichen und hartnäckigen Metastasen in der Lunge.
        \end{dialog}

        \begin{dialog}{Patrick}
            Und, äh, wie geht es dir heute?
        \end{dialog}

        \begin{dialog}[In Gedanken]{Hazel}
            Du meinst abgesehen von meinem tödlichen Krebs?
        \end{dialog}

        \begin{decision}{wie-geht-es-hazel}{Wie geht es \gls{Hazel}?}{\gls{Hazel}}
            \begin{option}{hazel-geht-es-gut}{Ganz gut.}
                \begin{dialog}{Hazel}
                    Ganz gut.
                \end{dialog}
            \end{option}
            \begin{option}{hazel-geht-es-nicht-gut}{Naja, ich habe halt Krebs.}
                \begin{dialog}{Hazel}
                    Naja, wie es einem halt so geht, wenn man Krebs hat.
                \end{dialog}
            \end{option}
            \begin{option}{hazel-geht-es-schlecht}{Schlecht.}
                \begin{dialog}{Hazel}
                    Ganz ehrlich: Nicht gut.
                    Aber das brauche ich hier, glaube ich, keinem zu erzählen.
                \end{dialog}

                \begin{decision}{patrick-bittet-hazel-über-ihre-gefühle-zu-erzählen}{\gls{Patrick} bittet \gls{Hazel} über ihre Gefühle zu erzählen.}{\gls{Patrick}}
                    \begin{option}{patrick-lässt-hazel-erzählen}{Oh doch, erzähl.}
                        \begin{dialog}{Patrick}
                            Oh doch, erzähl.
                            Dafür sind wir hier!
                        \end{dialog}

                        \begin{dialog}{Hazel}
                            Es ist wie eine Behinderung.
                            Immer muss ich diese Flasche mitschleppen.
                            Ich kann keine längere Strecke gehen, ohne wie Darth Vader zu atmen und wenn ich eine Treppe hochgehe, muss ich mich erstmal 5 Minuten ausruhen.
                            Aber immer, wenn ich mich darüber beschwere und hier andere sehe, wie \gls{Isaac}, der bald sein Augenlicht verliert, dann habe ich irgendwie das Gefühl, dass ich noch halbwegs glücklich sein kann.
                        \end{dialog}

                        \begin{dialog}{Patrick}
                            \gls{Hazel}, wir sind für dich da!
                        \end{dialog}

                        \begin{dialog}{Alle im Chor}
                            Wir sind für dich da \gls{Hazel}.
                        \end{dialog}
                    \end{option}

                    \begin{option}{patrick-lässt-hazel-nicht-erzählen}{Wir sind für dich da, \gls{Hazel}.}
                        \begin{dialog}{Patrick}
                            \gls{Hazel}, wir sind für dich da!
                        \end{dialog}

                        \begin{dialog}{Alle im Chor}
                            Wir sind für dich da \gls{Hazel}.
                        \end{dialog}
                    \end{option}
                \end{decision}
            \end{option}
        \end{decision}

        \begin{dialog}{Augustus}
            Nun noch der neue in der Runde, Augustus heißt du, richtig?
        \end{dialog}

        \begin{dialog}{Augustus}
            Ja genau, ich heiße Augustus.
            Augustus Waters.
            Ich bin siebzehn.
            Vor anderthalb Jahren hatte ich den leichten Anflug eines Osteosarkoms, also Knochenkrebs.
            Daher habe ich auch diese Beinprothese...

            \actionline{\gls{Augustus} krempelt sein Hosenbein hoch, sodass seine Prothese zu sehen ist.}

            So kann man ganz einfach ein paar Kilo abnehmen!

            Aber heute bin ich hier, weil \gls{Isaac} mich darum gebeten hat.
        \end{dialog}

        \begin{dialog}{Patrick}
            Und wie geht's dir, Gus?
        \end{dialog}

        \begin{dialog}{Augustus}
            Prächtig, prächtig.
            Meine Achterbahn geht nur nach oben!
        \end{dialog}

        \begin{dialog}[IN Gedanken]{Hazel}
            Warum starrt mich Augustus die ganze Zeit an?
            Ich meine, er ist ziemlich süß, aber warum ich?
            Es gibt doch so viele schöne Mädchen in der Welt!
        \end{dialog}

        \begin{dialog}{Patrick}
            \gls{Augustus}, vielleicht möchtest du der Gruppe von deinen Ängsten erzählen?
        \end{dialog}

        \begin{dialog}{Augustus}
            Meine Ängste?
            Ich habe Angst vor dem Vergessen.
            Ich fürchte das Vergessen wie der sprichwörtliche Blinde, der die Dunkelheit fürchtet.
        \end{dialog}

        \begin{decision}{selbsthilfegruppe-hazel-meldet-sich}{\gls{Hazel} spricht über das Vergessen}{\gls{Hazel}}
            \begin{option}{selbsthilfegruppe-sich-melden}{\gls{Hazel} meldet sich.}
                \actionline{\gls{Hazel} hebt die Hand.}

                \begin{dialog}{Patrick}
                    Oh, \gls{Hazel}, das kommt aber überraschend!
                \end{dialog}

                \begin{dialog}{Hazel}
                    Augustus, es kommt eine Zeit, da wir alle tot sind.
                    Wir alle.
                    Es kommt die Zeit, da es keine Menschen mehr gibt, die sich erinnern können, dass je irgendwer von uns existiert hat oder dass unsere Spezies je etwas geleistet hat.
                    Dann ist keiner mehr da, der sich an Aristoteles oder Kleopatra erinnert und erst recht nicht an dich.
                    Alles, was wir getan oder gebaut, geschrieben, gedacht oder entdeckt haben, alles wird vergessen sein, und all das hier...

                    \actionline{\gls{Hazel} macht eine allumfassende Geste.}

                    ...hat keine Bedeutung mehr.
                    Vielleicht kommt diese Zeit bald, vielleicht erst in Millionen von Jahren, aber selbst, wenn wir den Kollaps unserer Sonne überleben sollten, überleben wir nicht für immer.
                    Es gab eine Zeit, bevor die Organismen zu Bewusstsein kamen, und es wird eine Zeit danach geben.
                    Und wenn es die Unausweichlichkeit des menschlichen Vergessens ist, die dir Angst macht, dann rate ich dir eins: ignorier‘ sie einfach.
                    Das ist, weiß Gott, was alle anderen machen.
                \end{dialog}

                \actionline{Es entsteht eine kurze Pause, \gls{Augustus} lächelt \gls{Hazel} an.}

                \begin{dialog}{Patrick}
                    Hazel, das, was du gesagt hast, ist echt gut, du solltest dich öfters melden!

                    Lasst uns daraufhin beten!
                \end{dialog}
            \end{option}
            \begin{option}{selbsthilfegruppe-sich-nicht-melden}{\gls{Hazel} meldet sich nicht.}
                \actionline{Es entsteht eine kurze Pause.}

                \begin{dialog}{Patrick}
                    Nun, ich bin kein Pfarrer, aber ich glaube, dass wir nie vergessen werden.
                    Gott wird uns ewig an Vergangenes erinnern, selbst wenn wir es vergessen.
                    Ich würde dir daher empfehlen, deine Angst einfach zu ignorieren.

                    Lasst uns dafür nochmal beten.
                \end{dialog}
            \end{option}
        \end{decision}

        \begin{dialog}[betend]{Patrick}
            Herr Jesus Christus, als Krebspatienten haben wir uns hier in deinem Herzen versammelt, deinem buchstäblichen Herzen.
            Du, und du allein, kennst uns, wie wir uns selbst kennen.
            Führe uns durch die Zeiten der Prüfungen zum Leben und zum Licht.
            Wir beten für Isaacs Augen, für Michaels und Jamies Blut, für Augustus‘ Knochen, für Hazels Lunge und für James‘ Luftröhre.
            Wir beten, dass du uns heilen mögest und dass wir deine Liebe spüren und deinen Frieden, der über jedes Verständnis hinausgeht.
            Und wir erinnern uns im Herzen an die, die wir kannten und liebhatten und die heim zu dir gegangen sind: Maria und Kade und Joseph und Haley und Abigail und Angelina und Taylor und Gabriel...
        \end{dialog}

        \begin{dialog}[In Gedanken]{Hazel}
            Oh Gott, das ist eine lange Liste.
            Auf der Welt gibt es eine Menge Tote.
            Eines Tages werde ich auch auf der Liste landen.
            Ganz am Ende, wenn keiner mehr zuhört.
        \end{dialog}

        \begin{dialog}{Patrick}
            Ok, Leute, jetzt noch unser Mantra:
        \end{dialog}

        \begin{dialog}{Alle im Chor}
            Unser bestes Leben heute Leben.
        \end{dialog}

        \begin{dialog}{Patrick}
            Ok, dann sehen wir uns nächste Woche!
        \end{dialog}
    \end{scene}

    \begin{scene}{aussen}{Holländisches Picknick}{Nachmittags}
        \begin{scenedescription}
            \gls{Hazel} und \gls{Augustus} betreten den Virginia B. Fairbanks Art \& Nature Park.
            In dem Park hat der Künstler Joep van Lieshout sein Kunstwerk "Funky bones" ausgestellt.
            Das Kunstwerk besteht aus 20 Sitzbänken ohne Lehne, welche in Form eines menschlichen Skeletts angeordnet und auch entsprechend bemalt sind.
            \gls{Hazel} und \gls{Augustus} betreten den Park über einen Kiesweg.
            \gls{Augustus} hat einen Picknick-Korb in der Hand, in dem Tomaten-Käse-Sandwiches sind.
        \end{scenedescription}

        \begin{dialog}{Hazel}
            Wow, ist das ein wundervoller Tag!
        \end{dialog}

        \begin{dialog}{Augustus}
            Ja!
        \end{dialog}

        \begin{dialog}{Hazel}
            Gehst du mit allen deinen Eroberungen in diesen Park?
        \end{dialog}

        \begin{dialog}{Augustus}
            Ja, klar.
            Das ist Standard.
            Und wahrscheinlich auch der Grund, warum ich noch Jungfrau bin!
        \end{dialog}

        \begin{dialog}{Hazel}[erstaunt]
            Du bist doch keine Jungfrau mehr?
        \end{dialog}

        \begin{dialog}{Augustus}
            Komm her, ich zeig' dir mal was.
            Siehst du diesen Kreis?
        \end{dialog}

        \actionline{\gls{Augustus} stellt den Picknick-Korb ab und hebt einen Stock vom Boden auf.
        Mit dem Stock zeichnet er einen Kreis in den Kies.}

        \begin{dialog}{Augustus}
            Das hier ist die Menge aller Jungfrauen!
            Und das...
        \end{dialog}

        \actionline{\gls{Augustus} zeichnet einen kleineren Kreis innerhalb des großen Kreises.}

        \begin{dialog}{Augustus}[Fortsetzung]
            ...sind die 18-jährigen Einbeinigen.
        \end{dialog}

        \actionline{\gls{Augustus} legt den Stock auf den Boden und nimmt den Picknick-Korb wieder in die Hand.}

        \begin{dialog}{Augustus}
            Naja.
        \end{dialog}

        \actionline{\gls{Augustus} und \gls{Hazel} gehen weiter in richtung von Funky bones.}

        \begin{dialog}{Augustus}
            Funky bones, von Joep van Lieshout.
        \end{dialog}

        \begin{dialog}{Hazel}[lächelnd]
            Klingt nach Holländer.
        \end{dialog}

        \begin{dialog}{Augustus}
            Und das ist er auch, genau so wie Rick Smith und Tulpen.
        \end{dialog}
    \end{scene}

    \begin{scene}{innen}{Verhör (Fortsetzung)}{Tageszeit unbekannt}
        \begin{dialog}{Emma}
            Ich war es nicht.
            Er war es!
        \end{dialog}

        \begin{dialog}{Polizist}
            Und wer ist „er“?
        \end{dialog}

        \actionline{Emma spielt mit ihren Händen (Beat).}

        \begin{dialog}{Emma}
            Wenn ich Ihnen das sage, wird er Sie töten.
        \end{dialog}

        \begin{dialog}{Polizist}
            Hier sage ich etwas sehr Langes, damit der Zeilenumbruch im Dialog sichtbar wird.
            Das ist echt schwierig, denn es soll ja einen Inhalt haben, aber gleichzeitig bin ich sehr unkreativ.
            Technische Dinge eben.
            Vielleicht sollte ich eher einen Lorem Ipsum Text kopieren.
        \end{dialog}

        \begin{decision}{erste-entscheidung}{Eine Beispielentscheidung}{Emma}
            \begin{option}{test-label}{Test}
                \begin{dialog}{Emma}
                    Hier spreche ich eine Option.
                    Die Option ist sehr lange, weil auch ich einen Zeilenumbruch testen soll.
                    Haha, wie einfach.
                \end{dialog}
            \end{option}
            \begin{option}{test-label-other}{Eine weitere Option}
                \begin{dialog}{Emma}
                    Und hier die zweite Option.
                \end{dialog}
            \end{option}
        \end{decision}

        \begin{decision}{zweite-entscheidung}{Eine weitere Beispielentscheidung}{Polizist}
            \begin{option}{more-test-label}{Test}
                \begin{dialog}{Emma}
                    Nächste Entscheidung, hier ist die erste Option.
                \end{dialog}
            \end{option}
            \begin{option}{more-test-label-other}{Eine weitere Option}
                \begin{dialog}{Emma}
                    Und hier die zweite Option der zweiten Entscheidung.
                \end{dialog}
            \end{option}
        \end{decision}

        \begin{conditional}{erste-entscheidung}{test-label}
            \begin{dialog}{Polizist}
                Leider haben Sie die falsche option bei Entscheidung~\ref{erste-entscheidung} gewählt.
            \end{dialog}
        \end{conditional}

    \end{scene}

    \begin{scene}{innen}{Verschachtelte Entscheidungen}{Tageszeit unbekannt}
        \begin{scenedescription}
            In dieser Szene geht es um die Darstellung verschachtelter Entscheidungen.
        \end{scenedescription}

        \begin{dialog}{Emma}
            Auch hier sage ich zunächst etwas sehr langes, um die Umbrüche und die Einrückung zu testen.
            Viel Spaß beim Lesen!
        \end{dialog}

        \begin{decision}{obere-ebene-verschachtelte-entscheidung}{Eine verschachtelte Entscheidung -- obere Ebene}{Emma}
            \begin{option}{nested-option-1}{Erste Option}
                \begin{decision}{erste-innere-entscheidung}{Hier die erste verschachtelte Option - innen}{Polizist}
                    \begin{option}{innen-entscheidung-1-option1}{Innen ENtscheidung 1, Option 1}
                        \begin{dialog}{Emma}
                            Innere Entscheidung, hier ist die erste Option.
                            Der Text ist hier absichtlich lang, da jede Option aktuell weiter eingerückt ist, als die vorherige.
                            Das sollte so nicht sein, Dialog sollte immer die gleiche Einrückung haben.
                        \end{dialog}
                    \end{option}
                    \begin{option}{innen-entscheidung-1-option2}{Eine weitere innere Option}
                        \begin{dialog}{Emma}
                            Und hier die zweite Option der ersten inneren Entscheidung.
                        \end{dialog}
                    \end{option}
                \end{decision}
            \end{option}
            \begin{option}{nested-option-2}{Eine weitere äußere Option}
                \begin{decision}{zweite-innere-entscheidung}{Hier die zweite verschachtelte Option - innen}{Polizist}
                    \begin{option}{innen-entscheidung-2-option1}{Innen ENtscheidung 2, Option 1}
                        \begin{dialog}{Emma}
                            Zweite innere Entscheidung, hier ist die erste Option.
                        \end{dialog}
                    \end{option}
                    \begin{option}{innen-entscheidung-2-option2}{Eine weitere innere Option}
                        \begin{dialog}{Emma}
                            Und hier die zweite Option der zweiten inneren Entscheidung.
                        \end{dialog}
                    \end{option}
                \end{decision}
            \end{option}
        \end{decision}
    \end{scene}

\end{document}
