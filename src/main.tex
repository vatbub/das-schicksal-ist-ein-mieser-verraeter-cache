%! suppress = Ellipsis
% Preamble
\documentclass[12pt]{article}

\usepackage[utf8]{inputenc}
\usepackage[T1]{fontenc}

% Translation
\newcommand{\translationDefaultCharacterDescription}{Keine Charakterbeschreibung angegeben.}
\newcommand{\translationDecisionLabel}{Entscheidung}
\newcommand{\translationDecisionPrincipalSuffix}{trifft die Entscheidung}
\newcommand{\translationOptionLabel}{Option}
\newcommand{\translationEndOfBranch}{Ende von}
\newcommand{\translationConditionalLabel}{Bedingung}
\newcommand{\translationBeginningOfConditional}{Wenn}
\newcommand{\translationContinued}{Fortsetzung}

\newcommand{\translationWrittenBy}{Geschrieben von:}
\newcommand{\translationBasedOn}{Basierend auf:}


% Read preamble
% Document font
\usepackage{courier}

% Margins
\usepackage[a4paper, top=2.5cm, bottom=2cm, left=2.5cm, right=2.5cm]{geometry}
\setlength{\parindent}{0pt}
\setlength{\parskip}{\baselineskip}

% Packages
\usepackage{glossaries}
\usepackage{changepage}
\usepackage{xparse}
\usepackage{titling}
\usepackage{array}
\usepackage{enumerate}
\usepackage[shortlabels]{enumitem}
\usepackage{titlesec}
\usepackage{calc}

% For title slug
\newcolumntype{P}[1]{>{\centering\arraybackslash}p{#1}}

% Section header formatting
%! suppress = MissingLabel
\titleformat{\section}{}{}{0em}{\MakeUppercase}

\makeglossaries

% Appearance of characters
\newcommand{\character}[2][\translationDefaultCharacterDescription]{\newacronym{#2}{#2}{#1}}
\newacronymstyle{screenplay-characters}
{%
    \GlsUseAcrEntryDispStyle{short-long}%
}
{%
    \GlsUseAcrStyleDefs{short-long}%
    \renewcommand{\firstacronymfont}[1]{\MakeUppercase{##1}}%
    %
    % No case change, singular first use:
    \renewcommand*{\genacrfullformat}[2]{%
        \firstacronymfont{\glsentryshort{##1}}%
    }%
    % First letter upper case, singular first use:
    \renewcommand*{\Genacrfullformat}[2]{%
        \firstacronymfont{\Glsentryshort{##1}}%
    }%
    % No case change, plural first use:
    \renewcommand*{\genplacrfullformat}[2]{%
        \firstacronymfont{\glsentryshortpl{##1}}%
    }%
    % First letter upper case, plural first use:
    \renewcommand*{\Genplacrfullformat}[2]{%
        \firstacronymfont{\Glsentryshortpl{##1}}%
    }%
}

\setacronymstyle{screenplay-characters}

% Elements of a scene
\newenvironment{scene}[4][\unskip]{
    \textbf{\uppercase{#1}}


    \section{#2 #3 -- #4}\label{sec:scene-#3}
    }{}
\newenvironment{scenedescription}{%
    \begin{flushleft}
    }{
    \end{flushleft}
}

\newcommand{\standaloneactionline}[1]{#1}
\newcommand{\actionline}[1]{\standaloneactionline{#1}}
\newcommand{\dialogactionline}[2]{
    \end{flushleft}
    \end{adjustwidth}
    #1
    \begin{adjustwidth}{6.5cm}{0cm}
        \uppercase{#2}~(\translationContinued)
    \end{adjustwidth}
    \begin{adjustwidth}{2.5cm}{0cm}
        \begin{flushleft}
}

% Dialog elements
\NewDocumentEnvironment{dialog} { o m } {
    \begin{adjustwidth}{6.5cm}{0cm}
        \uppercase{#2}\IfValueT{#1}{\uppercase{~(#1)}}
    \end{adjustwidth}
    \renewcommand{\actionline}[1]{\dialogactionline{##1}{#2}}
    \begin{adjustwidth}{2.5cm}{0cm}
        \begin{flushleft}
        }{
        \end{flushleft}
    \end{adjustwidth}
\renewcommand{\actionline}[1]{\standaloneactionline{#1}}
}

% Other helpful stuff
\newcommand{\direction}[1]{\emph{(#1)}}
\newcommand{\pause}[0]{\direction{Pause}}
\newcommand{\beat}[0]{\direction{Beat}}

% Decision elements
\newlist{optionlist}{enumerate}{3}
\setlist[optionlist]{
    font=\bfseries,
    align=left
}
\setlist[optionlist, 1]{
    label=\translationOptionLabel~\arabic*:,
    ref=\arabic*
}
\setlist[optionlist, 2]{
    label=\translationOptionLabel~\theoptionlisti.\arabic*:,
    ref=\theoptionlisti.\arabic*
}
\setlist[optionlist, 3]{
    label=\translationOptionLabel~\theoptionlisti.\theoptionlistii.\arabic*:,
    ref=\theoptionlisti.\theoptionlistii.\arabic*
}

\newcounter{DecisionEnvironmentDepth}
\setcounter{DecisionEnvironmentDepth}{0}
\newcounter{decision}
\newenvironment{decision}[3]{
    \refstepcounter{decision}
    \addtocounter{DecisionEnvironmentDepth}{1}% increment depth
    \expandafter\newcommand\csname DecisionLabel\the\value{DecisionEnvironmentDepth} \endcsname{#1}
    \label{#1}
    \par\textbf{\translationDecisionLabel~\thedecision:}~#2 (#3~\translationDecisionPrincipalSuffix)
    \begin{optionlist}
    }{
    \end{optionlist}
    \addtocounter{DecisionEnvironmentDepth}{-1}% decrement depth
}

\newlength{\enumerateIndent}
\setlength{\enumerateIndent}{-1cm}
\newcounter{OptionEnvironmentDepth}
\setcounter{OptionEnvironmentDepth}{0}
\newcommand{\currentOptionLabel}{}
\newenvironment{option}[2]{
    \renewcommand{\currentOptionLabel}{#1}
    \addtocounter{OptionEnvironmentDepth}{1}% increment depth
    \item \label{#1} #2
    \begin{adjustwidth}{\enumerateIndent}{0cm*\value{OptionEnvironmentDepth}}
    }{
    \end{adjustwidth}
    \textit{(\translationEndOfBranch~\translationDecisionLabel~\ref{\csname DecisionLabel\the\value{DecisionEnvironmentDepth} \endcsname},~\translationOptionLabel~\ref{\currentOptionLabel})}
    \addtocounter{OptionEnvironmentDepth}{-1}% decrement depth
}

\newcounter{ConditionalEnvironmentDepth}
\setcounter{ConditionalEnvironmentDepth}{0}
\newcounter{conditional}
\newenvironment{conditional}[2]{
    \refstepcounter{conditional}
    \addtocounter{ConditionalEnvironmentDepth}{1}% increment depth
    \expandafter\newcommand\csname ConditionalCounter\the\value{ConditionalEnvironmentDepth} \endcsname{\theconditional}
    \textbf{\translationConditionalLabel~\theconditional:} \translationBeginningOfConditional~\translationDecisionLabel~\ref{#1}, \translationOptionLabel~\ref{#2}:
        }{
    \textit{(\translationEndOfBranch~\translationConditionalLabel~\csname ConditionalCounter\the\value{ConditionalEnvironmentDepth} \endcsname.)}
    \addtocounter{ConditionalEnvironmentDepth}{-1}% decrement depth
}


% Characters
\character{Hazel}
\character{Augustus}
\character{Hazels Mutter}
\character{Dr. Jim}
\character{Patrick}
\character{Isaac}
\character{Monica}


% Title
\title{Das Schicksal ist ein mieser Verräter}
\author{Frederik Kammel}
\newcommand{\BasedOn}{Das Schicksal ist ein mieser Verräter, John Green}
\newcommand{\TitleSlug}{Stell dir vor, dein Leben liegt in Trümmern. \newline Der letzte Kuss ist Jahre her.}

% Document
\begin{document}
    \fontfamily{pcr}\selectfont

\begin{titlepage}
    \centering
    \vglue6cm
    „\MakeUppercase{\thetitle}“

    \vspace{2cm}
    \translationWrittenBy

    \theauthor

    \vspace{2cm}
    \translationBasedOn

    \BasedOn

    \ifdefined\TitleSlug
    \vspace{2cm}

    \begin{adjustwidth}{1cm}{1cm}
        \begin{center}
            \begin{tabular}{ P{13cm} }
                \hline
                \textit{\TitleSlug} \\
                \hline
            \end{tabular}
        \end{center}
    \end{adjustwidth}
    \fi
\end{titlepage}



    \begin{scene}[fade in]{innen}{Prolog}{Tagsüber}
        \begin{dialog}[V.O.]{Hazel}
            Depression ist eine Nebenwirkung von Krebs.
            So steht es zumindest in jeder Krebs-Broschüre oder auf jeder Website im Internet.
            In Wirklichkeit sind Depressionen aber eine Nebenwirkung des Sterbens.
            Krebs ist auch eine Nebenwirkung des Sterbens.

            \pause

            Meine Mutter war der festen Überzeugung, dass ich eine Therapie brauchte, denn ich saß die ganze Zeit in
            meinem Zimmer, verließ kaum das Haus, las immer wieder das selbe Buch, aß wenig und verbrachte den großen
            Teil meiner reichlichen Zeit, um über den Tod nachzudenken.
            So gingen wir zu \gls{Dr. Jim}, der meiner Mutter bestätigte, dass ich bis zum Hals in einer lähmenden und absolut
            klinischen Depression steckte und dass meine Medikamente neu eingestellt werden müssten und ich außerdem
            einmal die Woche eine Selbsthilfegruppe besuchen sollte.
        \end{dialog}
    \end{scene}

    \begin{scene}[cut to]{innen}{Selbsthilfegruppe 1}{Nachmittag}
        \begin{scenedescription}
            Eine Gruppe Jugendlicher sitzt in einem Stuhlkreis in einem Kellerraum einer Kirche, unter ihnen ein älterer Betreuer, \gls{Patrick}.
            In der Mitte des Stuhlkreises ist ein Teppich ausgerollt, auf dem eine Abbildung Jesu-Christi zu sehen ist.
            Auf der Abbildung ist sein Herz deutlich zu sehen.
            Unter den Jugendlichen sitzen:

            \gls{Hazel}.
            Sie hat Lungenkrebs und zieht daher ständig eine Sauerstoffflasche hinter sich her.

            \gls{Isaac}.
            Er hat Augenkrebs, weswegen bei ihm ein Auge durch ein Glasauge ersetzt wurde.

            \gls{Augustus}.
            Er hatte Knochenkrebs.
            Im Rahmen seiner Behandlung wurde ihm sein rechtes Bein amputiert, weshalb er dort eine Prothese trägt.

            Die Stimmung ist sehr deprimierend, weswegen jeder nur nach explizitem Aufruf von \gls{Patrick} spricht, ansonsten ist es still.
        \end{scenedescription}

        \actionline{\gls{Augustus} starrt die ganze Zeit \gls{Hazel} an.}

        \begin{dialog}[betend]{Patrick}
            Gott, gib mir die Gelassenheit, die Dinge hinzunehmen, die ich nicht ändern kann, den Mut, die Dinge zu ändern,
            die ich ändern kann, und die Weisheit, das eine vom anderen zu unterscheiden.
        \end{dialog}

        \begin{dialog}[Normal]{Patrick}
            Nun, wir sind alle hier, buchstäblich im Herzen Jesu, weil jeder von uns eine Krebsgeschichte hat.
            Ich zum Beispiel hatte als Kind Hodenkrebs.
            Meine Eltern und meine Ärzte dachten, ich würde von ihnen gehen.

            Aber Gott hat mich gerettet und so bin ich nun hier, bei euch.
            Um euch zu helfen.

            Wir haben ein paar neue Gesichter hier, vielleicht stellt ihr euch alle mal vor.
            Isaac, möchtest du heute anfangen?
            Ich weiß, dass dir in der kommenden Woche eine große Herausforderung bevorsteht.
        \end{dialog}

        \begin{dialog}{Isaac}
            Ok. \direction{Räuspert sich}

            Ich heiße \gls{Isaac}.
            Ich bin siebzehn.
            Ich habe Augenkrebs.
            Mir wurde ja schon ein Auge entfernt, aber am Montag werde ich wieder operiert und das zweite Auge wird herausgenommen.
            Danach bin ich blind, das ist klar.
            Ich will mich auch nicht beschweren oder so, denn viele von euch hat es ja viel schlimmer erwischt.
            Aber, na ja, blind werden ist auch irgendwie scheiße.
            Zum Glück habe ich aber eine verdammt heiße Freundin, \gls{Monica}.
            Weiß auch nicht, wie ich die verdient habe.

            Und ich habe tolle Freunde, wie \gls{Augustus} Waters, die mir beistehen.

            \actionline{\gls{Isaac} nickt \gls{Augustus} zu, der neben ihm sitzt.}

            Tja, so sieht's aus.
            Danke.
        \end{dialog}

        \begin{dialog}{Patrick}
            Wir sind für dich da, \gls{Isaac}.
            Sagen wir es ihm, Leute.
        \end{dialog}

        \begin{dialog}{Alle im Chor}
            Wir sind für dich da \gls{Isaac}.
        \end{dialog}

        \begin{dialog}{Patrick}
            Hazel, wie sieht es mit dir aus?
        \end{dialog}

        \begin{dialog}{Hazel}
            Oh, äh, ... Ich heiße Hazel.
            Ich bin sechzehn.
            Ursprünglich hatte ich Schilddrüsenkrebs, Stadium IV, aber mit umfänglichen und hartnäckigen Metastasen in der Lunge.
        \end{dialog}

        \begin{dialog}{Patrick}
            Und, äh, wie geht es dir heute?
        \end{dialog}

        \begin{dialog}[In Gedanken]{Hazel}
            Du meinst abgesehen von meinem tödlichen Krebs?
        \end{dialog}

        \begin{decision}{wie-geht-es-hazel}{Wie geht es \gls{Hazel}?}{\gls{Hazel}}
            \begin{option}{hazel-geht-es-gut}{Ganz gut.}
                \begin{dialog}{Hazel}
                    Ganz gut.
                \end{dialog}
            \end{option}
            \begin{option}{hazel-geht-es-nicht-gut}{Naja, ich habe halt Krebs.}
                \begin{dialog}{Hazel}
                    Naja, wie es einem halt so geht, wenn man Krebs hat.
                \end{dialog}
            \end{option}
            \begin{option}{hazel-geht-es-schlecht}{Schlecht.}
                \begin{dialog}{Hazel}
                    Ganz ehrlich: Nicht gut.
                    Aber das brauche ich hier, glaube ich, keinem zu erzählen.
                \end{dialog}

                \begin{decision}{patrick-bittet-hazel-über-ihre-gefühle-zu-erzählen}{\gls{Patrick} bittet \gls{Hazel} über ihre Gefühle zu erzählen.}{\gls{Patrick}}
                    \begin{option}{patrick-lässt-hazel-erzählen}{Oh doch, erzähl.}
                        \begin{dialog}{Patrick}
                            Oh doch, erzähl.
                            Dafür sind wir hier!
                        \end{dialog}

                        \begin{dialog}{Hazel}
                            Es ist wie eine Behinderung.
                            Immer muss ich diese Flasche mitschleppen.
                            Ich kann keine längere Strecke gehen, ohne wie Darth Vader zu atmen und wenn ich eine Treppe hochgehe, muss ich mich erstmal 5 Minuten ausruhen.
                            Aber immer, wenn ich mich darüber beschwere und hier andere sehe, wie \gls{Isaac}, der bald sein Augenlicht verliert, dann habe ich irgendwie das Gefühl, dass ich noch halbwegs glücklich sein kann.
                        \end{dialog}

                        \begin{dialog}{Patrick}
                            \gls{Hazel}, wir sind für dich da!
                        \end{dialog}

                        \begin{dialog}{Alle im Chor}
                            Wir sind für dich da \gls{Hazel}.
                        \end{dialog}
                    \end{option}

                    \begin{option}{patrick-lässt-hazel-nicht-erzählen}{Wir sind für dich da, \gls{Hazel}.}
                        \begin{dialog}{Patrick}
                            \gls{Hazel}, wir sind für dich da!
                        \end{dialog}

                        \begin{dialog}{Alle im Chor}
                            Wir sind für dich da \gls{Hazel}.
                        \end{dialog}
                    \end{option}
                \end{decision}
            \end{option}
        \end{decision}

        \begin{dialog}{Augustus}
            Nun noch der neue in der Runde, Augustus heißt du, richtig?
        \end{dialog}

        \begin{dialog}{Augustus}
            Ja genau, ich heiße Augustus.
            Augustus Waters.
            Ich bin siebzehn.
            Vor anderthalb Jahren hatte ich den leichten Anflug eines Osteosarkoms, also Knochenkrebs.
            Daher habe ich auch diese Beinprothese...

            \actionline{\gls{Augustus} krempelt sein Hosenbein hoch, sodass seine Prothese zu sehen ist.}

            So kann man ganz einfach ein paar Kilo abnehmen!

            Aber heute bin ich hier, weil \gls{Isaac} mich darum gebeten hat.
        \end{dialog}

        \begin{dialog}{Patrick}
            Und wie geht's dir, Gus?
        \end{dialog}

        \begin{dialog}{Augustus}
            Prächtig, prächtig.
            Meine Achterbahn geht nur nach oben!
        \end{dialog}

        \begin{dialog}[IN Gedanken]{Hazel}
            Warum starrt mich Augustus die ganze Zeit an?
            Ich meine, er ist ziemlich süß, aber warum ich?
            Es gibt doch so viele schöne Mädchen in der Welt!
        \end{dialog}

        \begin{dialog}{Patrick}
            \gls{Augustus}, vielleicht möchtest du der Gruppe von deinen Ängsten erzählen?
        \end{dialog}

        \begin{dialog}{Augustus}
            Meine Ängste?
            Ich habe Angst vor dem Vergessen.
            Ich fürchte das Vergessen wie der sprichwörtliche Blinde, der die Dunkelheit fürchtet.
        \end{dialog}

        \begin{decision}{selbsthilfegruppe-hazel-meldet-sich}{\gls{Hazel} spricht über das Vergessen}{\gls{Hazel}}
            \begin{option}{selbsthilfegruppe-sich-melden}{\gls{Hazel} meldet sich.}
                \actionline{\gls{Hazel} hebt die Hand.}

                \begin{dialog}{Patrick}
                    Oh, \gls{Hazel}, das kommt aber überraschend!
                \end{dialog}

                \begin{dialog}{Hazel}
                    Augustus, es kommt eine Zeit, da wir alle tot sind.
                    Wir alle.
                    Es kommt die Zeit, da es keine Menschen mehr gibt, die sich erinnern können, dass je irgendwer von uns existiert hat oder dass unsere Spezies je etwas geleistet hat.
                    Dann ist keiner mehr da, der sich an Aristoteles oder Kleopatra erinnert und erst recht nicht an dich.
                    Alles, was wir getan oder gebaut, geschrieben, gedacht oder entdeckt haben, alles wird vergessen sein, und all das hier...

                    \actionline{\gls{Hazel} macht eine allumfassende Geste.}

                    ...hat keine Bedeutung mehr.
                    Vielleicht kommt diese Zeit bald, vielleicht erst in Millionen von Jahren, aber selbst, wenn wir den Kollaps unserer Sonne überleben sollten, überleben wir nicht für immer.
                    Es gab eine Zeit, bevor die Organismen zu Bewusstsein kamen, und es wird eine Zeit danach geben.
                    Und wenn es die Unausweichlichkeit des menschlichen Vergessens ist, die dir Angst macht, dann rate ich dir eins: ignorier‘ sie einfach.
                    Das ist, weiß Gott, was alle anderen machen.
                \end{dialog}

                \actionline{Es entsteht eine kurze Pause, \gls{Augustus} lächelt \gls{Hazel} an.}

                \begin{dialog}{Patrick}
                    Hazel, das, was du gesagt hast, ist echt gut, du solltest dich öfters melden!

                    Lasst uns daraufhin beten!
                \end{dialog}
            \end{option}
            \begin{option}{selbsthilfegruppe-sich-nicht-melden}{\gls{Hazel} meldet sich nicht.}
                \actionline{Es entsteht eine kurze Pause.}

                \begin{dialog}{Patrick}
                    Nun, ich bin kein Pfarrer, aber ich glaube, dass wir nie vergessen werden.
                    Gott wird uns ewig an Vergangenes erinnern, selbst wenn wir es vergessen.
                    Ich würde dir daher empfehlen, deine Angst einfach zu ignorieren.

                    Lasst uns dafür nochmal beten.
                \end{dialog}
            \end{option}
        \end{decision}

        \begin{dialog}[betend]{Patrick}
            Herr Jesus Christus, als Krebspatienten haben wir uns hier in deinem Herzen versammelt, deinem buchstäblichen Herzen.
            Du, und du allein, kennst uns, wie wir uns selbst kennen.
            Führe uns durch die Zeiten der Prüfungen zum Leben und zum Licht.
            Wir beten für Isaacs Augen, für Michaels und Jamies Blut, für Augustus‘ Knochen, für Hazels Lunge und für James‘ Luftröhre.
            Wir beten, dass du uns heilen mögest und dass wir deine Liebe spüren und deinen Frieden, der über jedes Verständnis hinausgeht.
            Und wir erinnern uns im Herzen an die, die wir kannten und liebhatten und die heim zu dir gegangen sind: Maria und Kade und Joseph und Haley und Abigail und Angelina und Taylor und Gabriel...
        \end{dialog}

        \begin{dialog}[In Gedanken]{Hazel}
            Oh Gott, das ist eine lange Liste.
            Auf der Welt gibt es eine Menge Tote.
            Eines Tages werde ich auch auf der Liste landen.
            Ganz am Ende, wenn keiner mehr zuhört.
        \end{dialog}

        \begin{dialog}{Patrick}
            Ok, Leute, jetzt noch unser Mantra:
        \end{dialog}

        \begin{dialog}{Alle im Chor}
            Unser bestes Leben heute Leben.
        \end{dialog}

        \begin{dialog}{Patrick}
            Ok, dann sehen wir uns nächste Woche!
        \end{dialog}

        \actionline{Alle stehen auf.}
    \end{scene}

    \begin{scene}[cut to]{aussen}{Nach der Selbsthilfegruppe}{Nachmittag}
        \begin{scenedescription}
            \gls{Hazel} geht aus der Kirche hinaus und wartet vor der Tür darauf, von ihrer Mutter abgeholt zu werden.
            \gls{Augustus} stellt sich zu ihr dazu.
        \end{scenedescription}

        \begin{dialog}{Augustus}
            Buchstäblich im Herzen Jesu...
            Ich dachte wir sind in einem Kirchenkeller, aber wir sind buchstäblich im Herzen Jesu.
        \end{dialog}

        \actionline{In der Ferne hört man \gls{Isaac} und seine Freundin \gls{Monica} wie sie sich küssen.
        Außerdem knetet \gls{Isaac} \gls{Monica}s Brüste ziemlich fest.
        Zwischen den Küssen sagen sie abwechselnd:}

        \begin{dialog}[In der Ferne]{Isaac und Monica abwechselnd}
            Für immer!
        \end{dialog}

        \begin{dialog}{Hazel}
            Tut das nicht weh an den Möpsen?
        \end{dialog}

        \begin{dialog}{Augustus}
            Ja, schwer zu sagen, ob er sie erregen oder eine Brustuntersuchung durchführen will.
        \end{dialog}

        \begin{dialog}{Hazel}
            Und was soll das mit diesem "Für immer"?
        \end{dialog}

        \begin{dialog}{Augustus}
            Das ist irgendwie ihr großes Ding.
            Sie lieben sich \emph{für immer}, oder so.
            Ich glaube sie haben sich die Worte \emph{für immer} circa eine Million Mal per SMS geschickt.
        \end{dialog}

        \actionline{\gls{Augustus} blickt auf Hazel und es entseht eine kurze Pause.}

        \begin{dialog}{Hazel}
            Was ist?
        \end{dialog}

        \begin{dialog}{Augustus}
            Du siehst schön aus.
            Ich sehe gerne schöne Menschen an und vor einer Weile habe ich beschlossen, dass ich mir die einfachen Freuden im Leben nicht mehr verkneifen werde.
        \end{dialog}

        \actionline{Augustus lächelt, es entsteht ein kurzes Schweigen.}

        \begin{conditional}{selbsthilfegruppe-hazel-meldet-sich}{selbsthilfegruppe-sich-melden}
            \begin{dialog}{Augustus}
                Erst recht in Anbetracht der Tatsache, die du so wunderbar ausgeführt hast, dass alles in Vergessenheit endet und so weiter.
            \end{dialog}
        \end{conditional}

        \begin{decision}{nach-selbsthilfegruppe-wie-sieht-hazel-aus}{Wie sieht \gls{Hazel} aus?}{\gls{Augustus}}
            \begin{option}{nach-selbsthilfegruppe-hazel-sieht-aus-wie-natalie-portman}{\gls{Hazel} sieht wie Natalie Portman aus.}
                \begin{dialog}{Augustus}
                    Du siehst aus wie die Milleniums-Natalie Portman in V wie Vendetta.
                \end{dialog}

                \begin{dialog}{Hazel}
                    Nie gesehen.
                \end{dialog}

                \begin{dialog}{Augustus}
                    Wirklich?
                    Bildhübsches Mädchen mit Kurzhaarschnitt und Abneigung gegen die Obrigkeit verliebt sich rettungslos in einen Jungen, der in Schwierigkeiten steckt.
                    Deine Autobiografie, soweit ich es sehe.
                \end{dialog}
            \end{option}
            \begin{option}{nach-selbsthilfegruppe-hazel-sieht-aus-wie-eva-green-aus}{\gls{Hazel} sieht wie Eva Green aus.}
                \begin{dialog}{Augustus}
                    Du siehst aus wie Eva Green aus James Bond - Casino Royale aus.
                \end{dialog}

                \begin{dialog}{Hazel}
                    Nie gesehen.
                \end{dialog}

                \begin{dialog}{Augustus}
                    Wirklich?
                    Du hast James Bond nie gesehen?
                    Und speziell dieses Bond-Girl!
                    Bildhübsches Mädchen mit schwarzen Haaren verliebt sich rettungslos in einen Helden, der in Schwierigkeiten steckt.
                    Deine Autobiografie, soweit ich es sehe.
                \end{dialog}
            \end{option}
        \end{decision}

        \begin{dialog}{Augustus}
            Du solltest dir den Film ansehen.
        \end{dialog}

        \begin{dialog}{Hazel}
            Okay, ich leihe ihn mir mal aus.
        \end{dialog}

        \begin{dialog}{Augustus}
            Nein.
            Mit mir.
            Bei uns zu Hause.
            Jetzt.
        \end{dialog}

        \begin{conditional}{nach-selbsthilfegruppe-wie-sieht-hazel-aus}{nach-selbsthilfegruppe-hazel-sieht-aus-wie-natalie-portman}
            \begin{dialog}{Hazel}
                Und was, wenn du ein Axtmörder bist?
            \end{dialog}

            \begin{dialog}{Augustus}
                Ja, das kann man nie ausschließen.
            \end{dialog}
        \end{conditional}

        \begin{conditional}{nach-selbsthilfegruppe-wie-sieht-hazel-aus}{nach-selbsthilfegruppe-hazel-sieht-aus-wie-eva-green-aus}
            \begin{dialog}{Hazel}
                Und was, wenn du ein Axtmörder bist?
                So wie dieser James Bond?
            \end{dialog}

            \begin{dialog}{Augustus}
                Haha, man merkt, dass du keine Ahnung von James Bond hast! \direction{zwinkert}
            \end{dialog}
        \end{conditional}

        \actionline{\gls{Augustus} greift in seine Brusttasche und zückt eine Schachtel Zigaretten.
        Er nimmt eine Zigarette heraus und steckt sie sich zwischen die Lippen.}

        \begin{dialog}[entsetzt]{Hazel}
            Ist das dein Ernst?
            Findest du das cool?
            Oh Gott, jetzt hast du alles kaputt gemacht!
        \end{dialog}

        \begin{dialog}{Augustus}
            Alles?
        \end{dialog}

        \begin{dialog}{Hazel}
            Das alles, wo ein Typ, der weder unattraktiv noch unintelligent noch sonst irgendwie unakzeptabel ist, mich anstarrt und sich über den falschen Gebrauch von ``buchstäblich'' lustig macht und mich mit Schauspielerinnen vergleicht und fragt, ob ich einen Film mit ihm sehen will.
            Es gibt immer eine Hamartie, oder?
            Und deine ist, dass du dir \direction{verärgert} obwohl du schonmal Krebs hattest einer Firma Geld bezahlst, um nochmal Krebs zu bekommen!
            Ich kann dir versichern:
            \direction{ausrufend}
            Nicht atmen zu können ist richtig scheiße!
        \end{dialog}

        \begin{dialog}{Augustus}
            Hamartie?
        \end{dialog}

        \begin{dialog}{Hazel}
            Ein tödlicher Fehler.
        \end{dialog}

        \actionline{\gls{Hazel} wendet sich von \gls{Augustus} ab und geht zum Bordstein.}

        \begin{dialog}[lächelnd]{Augustus}
            Weißt du, sie bringen einen nur um, wenn man sie anzündet.
            Es ist eine Metapher: Du steckst dir das tödliche Ding in den Mund, gibst ihm aber nicht die Kraft, dich zu töten.
        \end{dialog}

        \begin{dialog}[argwöhnisch]{Hazel}
            Eine Metapher?
        \end{dialog}

        \begin{dialog}{Augustus}
            Eine Metapher.
        \end{dialog}

        \actionline{\gls{Hazels Mutter} kommt mit dem Auto an, um \gls{Hazel} abzuholen.}

        \begin{dialog}{Mutter}
            Hey Mäuschen, Top Model wartet!
        \end{dialog}

        \begin{dialog}{Hazel}
            Nein Mom, ich werde mir mit A\gls{Augustus} Waters einen Film ansehen.
        \end{dialog}
    \end{scene}

    \begin{scene}{aussen}{Holländisches Picknick}{Nachmittags}
        \begin{scenedescription}
            \gls{Hazel} und \gls{Augustus} betreten den Virginia B. Fairbanks Art \& Nature Park.
            In dem Park hat der Künstler Joep van Lieshout sein Kunstwerk ``Funky bones'' ausgestellt.
            Das Kunstwerk besteht aus 20 Sitzbänken ohne Lehne, welche in Form eines menschlichen Skeletts angeordnet und auch entsprechend bemalt sind.
            \gls{Hazel} und \gls{Augustus} betreten den Park über einen Kiesweg.
            \gls{Augustus} in der einen Hand Einen Tulpenstrauß und in der anderen einen Picknick-Korb, in dem Tomaten-Käse-Sandwiches sind.
        \end{scenedescription}

        \begin{dialog}{Hazel}
            Wow, ist das ein wundervoller Tag!
        \end{dialog}

        \begin{dialog}{Augustus}
            Ja!
        \end{dialog}

        \begin{dialog}{Hazel}
            Gehst du mit allen deinen Eroberungen in diesen Park?
        \end{dialog}

        \begin{dialog}{Augustus}
            Ja, klar.
            Das ist Standard.
            Und wahrscheinlich auch der Grund, warum ich noch Jungfrau bin!
        \end{dialog}

        \begin{dialog}{Hazel}[erstaunt]
            Du bist doch keine Jungfrau mehr?
        \end{dialog}

        \begin{dialog}{Augustus}
            Komm her, ich zeig' dir mal was.
            Siehst du diesen Kreis?
        \end{dialog}

        \actionline{\gls{Augustus} stellt den Picknick-Korb ab und hebt einen Stock vom Boden auf.
        Mit dem Stock zeichnet er einen Kreis in den Kies.}

        \begin{dialog}{Augustus}
            Das hier ist die Menge aller Jungfrauen!
            Und das...
        \end{dialog}

        \actionline{\gls{Augustus} zeichnet einen kleineren Kreis innerhalb des großen Kreises.}

        \begin{dialog}{Augustus}[Fortsetzung]
            ...sind die 18-jährigen Einbeinigen.
        \end{dialog}

        \actionline{\gls{Augustus} legt den Stock auf den Boden und nimmt den Picknick-Korb wieder in die Hand.}

        \begin{dialog}{Augustus}
            Naja.
        \end{dialog}

        \actionline{\gls{Augustus} und \gls{Hazel} gehen weiter in Richtung von Funky bones.}

        \begin{dialog}{Augustus}
            Funky bones, von Joep van Lieshout.
        \end{dialog}

        \begin{dialog}{Hazel}[lächelnd]
            Klingt nach Holländer.
        \end{dialog}

        \begin{dialog}{Augustus}
            Und das ist er auch, genau so wie Rik Smits und Tulpen.
        \end{dialog}

        \actionline{\gls{Augustus} und \gls{Hazel} bleiben stehen. Augustus holt aus dem Picknick-Korb eine Picknick-Decke und breitet diese aus.
        Beide setzen sich auf die Decke und \gls{Augustus} holt die Sandwiches aus dem Korb.}

        \begin{decision}{hazel-fragt-nach-orange}{\gls{Hazel} fragt nach der Tulpenfarbe.}{Hazel}
            \begin{option}{hazel-fragt-nach-tulpenfarbe}{Warum denn orangene Tulpen?}
                \begin{dialog}{Hazel}
                    Wozu das ganze Orange?
                \end{dialog}

                \begin{dialog}{Augustus}
                    Die holländische Nationalfarbe, ist doch klar?
                \end{dialog}
            \end{option}

            \begin{option}{hazel-fragt-nicht-nach-tulpenfarbe}{Schweigen.}
                \actionline{\gls{Hazel} schweigt.}
            \end{option}
        \end{decision}

        \actionline{\gls{Augustus} hält \gls{Hazel} einen Sandwich hin.}

        \begin{dialog}{Augustus}
            Sandwich?
        \end{dialog}

        \begin{decision}{hazels-antwort-ob-sie-einen-sandwich-will}{\gls{Hazel}s Reaktion auf den Sandwich}{Hazel}
            \begin{option}{hazel-nimmt-den-sandwich-an}{Ja, gerne!}
                \begin{dialog}{Hazel}
                    Ja, gerne!
                \end{dialog}
            \end{option}

            \begin{option}{hazel-rfragt-ob-sandwiches-hollaendisch-sind}{Auch holländisch?}
                \begin{dialog}{Hazel}
                    Lass mich raten: Auch holländisch?
                \end{dialog}
            \end{option}
        \end{decision}

        \begin{dialog}{Augutus}
            Holländischer Käse mit Tomaten. Die Tomaten sind aber aus Mexiko. Tut mir Leid.
        \end{dialog}

        \actionline{\gls{Hazel} lacht etwas.}

        \begin{decision}{soll-hazel-einen-witz-ueber-tomaten-machen}{Witz über Tomaten}{Hazel}
            \begin{option}{einen-witz-ueber-tomaten-machen}{Wie wär's mit orangenen Tomaten?}
                \begin{dialog}{Hazel}
                    Hättest du nicht wenigstens orangene Tomaten besorgen können?
                \end{dialog}

                \actionline{\gls{Augustus} lacht und beide essen ihre Sandwiches weiter.}
            \end{option}

            \begin{option}{keinen-witz-machen}{Nichts sagen.}
                \actionline{\gls{Hazel} isst ihren Sandwich weiter.}
            \end{option}
        \end{decision}

        \actionline{\gls{Augustus} nimmt seine Zigarettenschachtel aus der Jackentasche, zieht sich daraus eine Zigarette heraus und steckt sie sich (unangezündet) zwischen die Lippen.}

        \begin{dialog}{Hazel}[In Gedanken]
            Ich frage mich echt, was Augustus mit all den holländischen Sachen machen will...
        \end{dialog}

        \actionline{Etwas von den beiden entfernt spielen einige Kinder und springen von Knochen zu Knochen.}

        \begin{dialog}{Augustus}
            Zwei Sachen, die ich an dieser Skulptur liebe:

            \actionline{Augustus nimmt die Zigarette zwischen die Finger und schnippt damit, als wollte er die Asche abklopfen. Dann steckt er sie sich wieder in den Mund.}

            Erstens sind die Knochen haargenau so weit voneinander entfernt, dass jedes Kind den unwiderstehlichen Drang verspürt, zwischen ihnen hin und her zu springen.
            Man muss einfach vom Brustkorb zum Schädel hüpfen und wieder zurück.
            Was bedeutet, das, zweitens, die Skulptur im Grunde Kinder dazu zwingt, auf Knochen zu spielen.
            Die symbolischen Deutungsmöglichkeiten sind unendlich, Hazel Grace.
        \end{dialog}

        % TODO: Mehrere Möglichkeiten geben, wie Hazel das Gespräch wieder auf Holland lenkt
        \begin{dialog}{Hazel}
            Du hast echt eine Schwäche für Symbole, was?
        \end{dialog}

        \begin{dialog}{Augustus}
            Stimmt.
            Wahrscheinlich fragst du dich zum Beispiel, warum du ein mieses Käsesandwich isst und Orangensaft trinkst und warum ich das Trikot eines Niederländers trage, der mal einen Sport gespielt hat, den ich zu hassen gelernt habe.
        \end{dialog}

        \begin{dialog}{Hazel}
            Tja, Fragen über Fragen...
        \end{dialog}

        \begin{dialog}{Augustus}
            Hazel Grace.
            Wie so viele Kinder vor dir - und das sage ich mit großer Zuneigung - hast du deinen Herzenswunsch vergeudet, ohne dir Gedanken über die Konsequenzen zu machen.
            Du hast dem Sensenmann ins Auge gesehen und die Angst, mit dem Wunsch in der sprichwörtlichen Tasche zu sterben, hat dich dazu verleitet, den ersten Wunsch zu äußern, der dir in den Sinn kam, und so hast du dich, wie viele andere vor und nach dir, für die kalten und künstlichen Freuden eines Freizeitparks entschieden.
        \end{dialog}

        \begin{dialog}{Hazel}
            Hey, das hat ziemlich Spaß gemacht!
            Ich habe Goofy kennengelernt und Minnie...
        \end{dialog}

        \begin{dialog}{Augustus}
            Ich bin mitten in einem bedeutenden Monolog!
            Du kannst mich da nicht einfach unterbrechen!

            Okay.
            Wo waren wir stehengeblieben?
        \end{dialog}

        \begin{dialog}{Hazel}
            Die künstlichen Freuden.
        \end{dialog}

        \actionline{\gls{Augustus} schiebt seine Zigarette zurück in das Päckchen.}

        \begin{dialog}{Augustus}
            Richtig.
            Die kalten und künstlichen Freuden eines Freizeitparks.
            Doch lass mich dir sagen, dass die wahren Helden der Wunschfabrik die jungen Männer und Frauen sind, die warten können wie alte Männer an der Bushaltestelle oder wie fromme Mädchen auf die Ehe.
            Stoisch und ohne zu klagen, warten sie darauf, dass ihnen ihr wahrer Herzenswunsch begegnet.
        \end{dialog}
    \end{scene}

    \begin{scene}{innen}{Verhör (Fortsetzung)}{Tageszeit unbekannt}
        \begin{dialog}{Emma}
            Ich war es nicht.
            Er war es!
        \end{dialog}

        \begin{dialog}{Polizist}
            Und wer ist „er“?
        \end{dialog}

        \actionline{Emma spielt mit ihren Händen (Beat).}

        \begin{dialog}{Emma}
            Wenn ich Ihnen das sage, wird er Sie töten.
        \end{dialog}

        \begin{dialog}{Polizist}
            Hier sage ich etwas sehr Langes, damit der Zeilenumbruch im Dialog sichtbar wird.
            Das ist echt schwierig, denn es soll ja einen Inhalt haben, aber gleichzeitig bin ich sehr unkreativ.
            Technische Dinge eben.
            Vielleicht sollte ich eher einen Lorem Ipsum Text kopieren.
        \end{dialog}

        \begin{decision}{erste-entscheidung}{Eine Beispielentscheidung}{Emma}
            \begin{option}{test-label}{Test}
                \begin{dialog}{Emma}
                    Hier spreche ich eine Option.
                    Die Option ist sehr lange, weil auch ich einen Zeilenumbruch testen soll.
                    Haha, wie einfach.
                \end{dialog}
            \end{option}
            \begin{option}{test-label-other}{Eine weitere Option}
                \begin{dialog}{Emma}
                    Und hier die zweite Option.
                \end{dialog}
            \end{option}
        \end{decision}

        \begin{decision}{zweite-entscheidung}{Eine weitere Beispielentscheidung}{Polizist}
            \begin{option}{more-test-label}{Test}
                \begin{dialog}{Emma}
                    Nächste Entscheidung, hier ist die erste Option.
                \end{dialog}
            \end{option}
            \begin{option}{more-test-label-other}{Eine weitere Option}
                \begin{dialog}{Emma}
                    Und hier die zweite Option der zweiten Entscheidung.
                \end{dialog}
            \end{option}
        \end{decision}

        \begin{conditional}{erste-entscheidung}{test-label}
            \begin{dialog}{Polizist}
                Leider haben Sie die falsche option bei Entscheidung~\ref{erste-entscheidung} gewählt.
            \end{dialog}
        \end{conditional}

    \end{scene}

    \begin{scene}{innen}{Verschachtelte Entscheidungen}{Tageszeit unbekannt}
        \begin{scenedescription}
            In dieser Szene geht es um die Darstellung verschachtelter Entscheidungen.
        \end{scenedescription}

        \begin{dialog}{Emma}
            Auch hier sage ich zunächst etwas sehr langes, um die Umbrüche und die Einrückung zu testen.
            Viel Spaß beim Lesen!
        \end{dialog}

        \begin{decision}{obere-ebene-verschachtelte-entscheidung}{Eine verschachtelte Entscheidung -- obere Ebene}{Emma}
            \begin{option}{nested-option-1}{Erste Option}
                \begin{decision}{erste-innere-entscheidung}{Hier die erste verschachtelte Option - innen}{Polizist}
                    \begin{option}{innen-entscheidung-1-option1}{Innen Entscheidung 1, Option 1}
                        \begin{dialog}{Emma}
                            Innere Entscheidung, hier ist die erste Option.
                            Der Text ist hier absichtlich lang, da jede Option aktuell weiter eingerückt ist, als die vorherige.
                            Das sollte so nicht sein, Dialog sollte immer die gleiche Einrückung haben.
                        \end{dialog}
                    \end{option}
                    \begin{option}{innen-entscheidung-1-option2}{Eine weitere innere Option}
                        \begin{dialog}{Emma}
                            Und hier die zweite Option der ersten inneren Entscheidung.
                        \end{dialog}
                    \end{option}
                \end{decision}
            \end{option}
            \begin{option}{nested-option-2}{Eine weitere äußere Option}
                \begin{decision}{zweite-innere-entscheidung}{Hier die zweite verschachtelte Option - innen}{Polizist}
                    \begin{option}{innen-entscheidung-2-option1}{Innen Entscheidung 2, Option 1}
                        \begin{dialog}{Emma}
                            Zweite innere Entscheidung, hier ist die erste Option.
                        \end{dialog}
                    \end{option}
                    \begin{option}{innen-entscheidung-2-option2}{Eine weitere innere Option}
                        \begin{dialog}{Emma}
                            Und hier die zweite Option der zweiten inneren Entscheidung.
                        \end{dialog}
                    \end{option}
                \end{decision}
            \end{option}
        \end{decision}
    \end{scene}

\end{document}
