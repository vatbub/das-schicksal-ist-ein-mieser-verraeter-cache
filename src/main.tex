% Preamble
\documentclass[12pt]{article}

\usepackage[utf8]{inputenc}
\usepackage[T1]{fontenc}

% Document font
\usepackage{courier}

% Margins
\usepackage[a4paper, top=2.5cm, bottom=2cm, left=2.5cm, right=2.5cm]{geometry}
\setlength{\parindent}{0pt}
\setlength{\parskip}{\baselineskip}

% Packages
\usepackage{glossaries}
\usepackage{changepage}
\usepackage{xparse}
\usepackage{titling}
\usepackage{array}
\usepackage{enumerate}
\usepackage[shortlabels]{enumitem}

% For title slug
\newcolumntype{P}[1]{>{\centering\arraybackslash}p{#1}}

\makeglossaries

% Appearance of characters
\newcommand{\character}[2][No character description specified.]{\newacronym{#2}{#2}{#1}}
\newacronymstyle{screenplay-characters}
{%
    \GlsUseAcrEntryDispStyle{short-long}%
}
{%
    \GlsUseAcrStyleDefs{short-long}%
    \renewcommand{\firstacronymfont}[1]{\MakeUppercase{##1}}%
    %
    % No case change, singular first use:
    \renewcommand*{\genacrfullformat}[2]{%
        \firstacronymfont{\glsentryshort{##1}}%
    }%
    % First letter upper case, singular first use:
    \renewcommand*{\Genacrfullformat}[2]{%
        \firstacronymfont{\Glsentryshort{##1}}%
    }%
    % No case change, plural first use:
    \renewcommand*{\genplacrfullformat}[2]{%
        \firstacronymfont{\glsentryshortpl{##1}}%
    }%
    % First letter upper case, plural first use:
    \renewcommand*{\Genplacrfullformat}[2]{%
        \firstacronymfont{\Glsentryshortpl{##1}}%
    }%
}

\setacronymstyle{screenplay-characters}

% Elements of a scene
\newenvironment{scene}[4][\unskip]{
    \textbf{\uppercase{#1}}


    \section{\uppercase{#2 #3 -- #4}}\label{sec:scene-#3}
    }{}
\newenvironment{scenedescription}{%
    \begin{flushleft}
    }{
    \end{flushleft}
}

\newcommand{\actionline}[1]{#1}

% Dialog elements
\NewDocumentEnvironment{dialog} { o m } {
    \begin{adjustwidth}{6.5cm}{0cm}
        \uppercase{#2}\IfValueT{#1}{\uppercase{~(#1)}}
    \end{adjustwidth}
    \begin{adjustwidth}{2.5cm}{0cm}
        \begin{flushleft}
        }{
        \end{flushleft}
    \end{adjustwidth}
}

Decision elements
\newcounter{decision}
\newenvironment{decision}[2]{
    \refstepcounter{decision}
    % \label{#1}
    \par\textbf{Entscheidung~\thedecision:}~#2
    \begin{enumerate}[label=\textbf{Option \arabic*:}]
    }{
    \end{enumerate}
}
\newenvironment{option}[2]{\item #2}{\textit{(Ende des Entscheidungsastes.)}}

\newenvironment{conditional}[2]{Wenn Entscheidung~\ref{#1}, Option~\ref{#2}:}{\textit{(Ende des optionalen Astes.)}}

% Characters
\character{Hazel}
\character{Augustus}
\character{Hazels Mutter}
\character{Dr. Jim}
\character{Patrick}
\character{Isaac}
\character{Monica}


% Title
\title{Das Schicksal ist ein mieser Verräter}
\author{Frederik Kammel}
\newcommand{\BasedOn}{Das Schicksal ist ein mieser Verräter, John Green}
\newcommand{\TitleSlug}{Stell dir vor, dein Leben liegt in Trümmern. \newline Der letzte Kuss ist Jahre her.}

% Document
\begin{document}

    \fontfamily{pcr}\selectfont

    \begin{titlepage}
        \centering
        \vglue6cm
        „\MakeUppercase{\thetitle}“

        \vspace{2cm}
        Geschrieben von:

        \theauthor

        \vspace{2cm}
        Basierend auf:

        \BasedOn

        \ifdefined\TitleSlug
        \vspace{2cm}

        \begin{adjustwidth}{1cm}{1cm}
            \begin{center}
                \begin{tabular}{ P{13cm} }
                    \hline
                    \textit{\TitleSlug} \\
                    \hline
                \end{tabular}
            \end{center}
        \end{adjustwidth}
        \fi
    \end{titlepage}


    \begin{scene}[fade in]{innen}{Verhör}{Tageszeit unbekannt}
        \begin{scenedescription}
            \gls{Emma}, eine 40 Jahre alte Frau mit faltigem Gesicht sitzt an einem Tisch.
            Ihr gegenüber sitzt ein \gls{Polizist}, auf dem Tisch stehen zwei Mikrofone, eines zeigt zum \gls{Polizist}en, eines zu Emma.
            Das Licht fällt so, dass der Tisch gut sichtbar und die Gesichter der Personen gerade so sichtbar sind.
            Der Rest des Raumes ist schwarz.
        \end{scenedescription}

        \begin{dialog}{Polizist}
            Sie wissen, warum Sie hier sind?
        \end{dialog}

        \actionline{\gls{Emma} schweigt.}

        \begin{dialog}[Fortsetzung]{Polizist}
            Sie müssen mir nichts sagen, aber wenn Sie schweigen, werden wir Sie bis auf Weiteres festnehmen müssen.
        \end{dialog}
    \end{scene}

    \begin{scene}{aussen}{Flashback}{tagsüber}
        \begin{scenedescription}
            \gls{Emma} liegt auf dem Boden und wird von einer anderen Person erwürgt.
            \gls{Emma} schreit und versucht, sich aus dem Würgegriff zu befreien, in ihrem Blick ist die Angst vor dem Tod deutlich zu sehen.
        \end{scenedescription}
    \end{scene}

    \begin{scene}{innen}{Verhör (Fortsetzung)}{Tageszeit unbekannt}
        \begin{dialog}{Emma}
            Ich war es nicht.
            Er war es!
        \end{dialog}

        \begin{dialog}{Polizist}
            Und wer ist „er“?
        \end{dialog}

        \actionline{Emma spielt mit ihren Händen (Beat).}

        \begin{dialog}{Emma}
            Wen ich Ihnen das sage, wird er Sie töten.
        \end{dialog}

        \begin{decision}{erste-entscheidung}{Eine Beispielentscheidung}
            \begin{option}{test-label}{Test}
                \begin{dialog}{Emma}
                    Hier spreche ich eine Option.
                \end{dialog}
            \end{option}
            \begin{option}{test-label-other}{Eine weitere Option}
                \begin{dialog}{Emma}
                    Und hier die zweite Option.
                \end{dialog}
            \end{option}
        \end{decision}

        \begin{decision}{zweite-entscheidung}{Eine weitere Beispielentscheidung}
            \begin{option}{more-test-label}{Test}
                \begin{dialog}{Emma}
                    Nächste Entscheidung, hier ist die erste Option.
                \end{dialog}
            \end{option}
            \begin{option}{more-test-label-other}{Eine weitere Option}
                \begin{dialog}{Emma}
                    Und hier die zweite Option der zweiten Entscheidung.
                \end{dialog}
            \end{option}
        \end{decision}

        \begin{conditional}{erste-entscheidung}{test-label}
            \begin{dialog}{Polizist}
                Leider haben Sie die falsche option bei Entscheidung~\ref{erste-entscheidung} gewählt.
                Wie wär's mit~\ref{#1} oder mit~\ref{test-label}?
            \end{dialog}
        \end{conditional}
    \end{scene}

    \begin{enumerate}
        \item First item
        \item \label{itm:bla-blub} Second item
    \end{enumerate}

    Hier reference~\ref{itm:bla-blub}

\end{document}
